% This is a simple LaTex sample document that gives a submission format
%   for IEEE PAMI-TC conference submissions.  Use at your own risk.

% Make two column format for LaTex 2e.
\documentclass[10pt,twocolumn]{article}
\usepackage{times}

% Use following instead for LaTex 2.09 (may need some other mods as well).
% \documentstyle[times,twocolumn]{article}

% Set dimensions of columns, gap between columns, and paragraph indent 
\setlength{\textheight}{8.875in}
\setlength{\textwidth}{6.875in}
\setlength{\columnsep}{0.3125in}
\setlength{\topmargin}{0in}
\setlength{\headheight}{0in}
\setlength{\headsep}{0in}
\setlength{\parindent}{1pc}
\setlength{\oddsidemargin}{-.1875in}  % Centers text.
\setlength{\evensidemargin}{-.1875in}

% Add the period after section numbers.  Adjust spacing.
\newcommand{\Section}[1]{\vspace{-8pt}\section{\hskip -1em.~~#1}\vspace{-3pt}} 
\newcommand{\SubSection}[1]{\vspace{-3pt}\subsection{\hskip -1em.~~#1}
     	\vspace{-3pt}}


\begin{document}

% Don't want date printed
\date{}

% Make title bold and 14 pt font (Latex default is non-bold, 16pt) 
\title{\Large\bf Xerxes : an advanced search system for your desktop}

% CVPR PAPER SUBMISSIONS MUST BE ANONYMOUS. DO NOT PUT YOUR NAME HERE.
% PLEASE PUT YOUR CVPR PAPER NUMBER IN THIS FIELD IF YOU KNOW IT.
% For single author (just remove % characters)
%\author{I. M. Anonymous \\
%  My Department \\
%  My Institute \\
%  My City, STATE, zip}

% CVPR PAPER SUBMISSIONS MUST BE ANONYMOUS. DO NOT PUT YOUR NAMES HERE.
% PLEASE PUT YOUR CVPR PAPER NUMBER IN THIS FIELD IF YOU KNOW IT.
% For two authors (default example)
\author{\begin{tabular}[t]{c@{\extracolsep{8em}}c} 
Claudia Tanase  & Razvan Alecsandrescu \\
 \\
        SPBA & SPBA \\
        UPB & UPB \\
        Bucharest, Romania & Bucharest, Romania 
\end{tabular}}

\maketitle


\section*{\centering Abstract}

{\em
This paper presents an advanced mechanism used to organize and seach files. The
system uses a combination of semantic labeling, file parsing and Virtual
Folders. }

\Section{Introduction}

One of the most important things for a computer user is to be able to organize
a large number of files and search for a specific one in a natural manner. 

Most of the current search applications can be used to find files using
information about the file name, file type or time/date information. 
But this only allows for simple searches. 

There is a lot to learn from web search engines like Google search engine. Not
only does it allow for simple searches but also for more complex searches:
including content analysis. 

This paper details the arhitecture of an advanced system for file
organization. Section 2 describes existing applications, Section 3
details the implementation of our system and Section 4 presents the results of
tests performed, conclusions and future work.

\SubSection{Previous Work}

Most of the searches that we can perform using standard OS tools are limited to
directory namespaces. For example in Windows we have the Search program that can
perform searches filtering the results by name,type,the time it was last modified, 
Unix shell wildcard patterns are useful for matching against
small numbers of files, and GNU locate for finding files in the directory hierarchy.
The biggest issue with these applications is that to perform queries on the
\textbf{data} inside a file can be a laborious procedure because of the linear
complexity invoved. Fast and reliable searches require an indexing structure.

To create more advanced search systems there are 2 aproaches that can be taken. 
The first one which is the most commonly used is to develop applications that
use the file system to obtain informations about files and then process, index
and allow the users to perform queries.

The second one is to implement \textit{file indexing inside the file system}. 

There are various pros and cons to each method of implementing a search system.
For example file indexing inside the file system assures that the index is
synchronized with the reality of the file system. The system assures us that
once a file is modified the changes will show up in the index. Also there can
be a small performance boost caused by the fact that processing a file that
already exists in the cache is fast. 

A application separated from the file system must keep track of all changes
(when files are moved, deleted, renamed, modified ) which is a little more
complex.




\Section{Summary and Conclusions}

This template will get you through the minimum article, i.e., with no
figures or equations.  To include those, please refer to your LaTeX
manual and the IEEE publications guidelines.  However, for a vision
conference you will probably want the following equation somewhere:
$$g(x) = {1\over\sqrt{2\pi}\sigma}e^{-x^2/2\sigma^2}$$
Good Luck!

% This is how to do an unnumbered section (note asterisk).
\section*{Acknowledgments}
bla bla bla
\begin{thebibliography}{9}
\small  % Use 9 point text.

% CVPR PAPER SUBMISSIONS MUST BE ANONYMOUS. DO NOT PUT YOUR NAME(S)
% IN ANY OF THE REFERENCES.

\bibitem{key:FOSMLV}
Mukund Gunti, Mark Pariente, Ting-Fang Yen, Stefan Zickler,
{\em File Organization and Search using Metadata, Labels, and Virtual Folders},
December 15, 2005
\bibitem{foo:baz}
Srinath Sridhar, Jeffrey Stylos and Noam Zeilberger,\\
{\em A Searchable-by-Content File System,} May 11, 2005

\end{thebibliography}
\end{document}